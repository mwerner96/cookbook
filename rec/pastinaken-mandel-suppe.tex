\section{Pastinaken-Mandel-Suppe mit Pumpernickelbrösel}
Pastinaken schmecken toll und machen sich immer gut in einer Suppe. Ich mag sie als Suppe fast lieber als Kartoffeln.

Das Pastinaken-Süppchen wird besonders cremig und bekommt durch die Zugabe von Madelmuss auch ein feines Mandelaroma. Die etwas herben Pumpernickel-Brösel fangen den leicht süßlichen Geschmack der Suppe gut auf. Das harmoniert sehr gut miteinander. Optisch ist sie auch sehr schön, wie ich finde.

Die Anregung zum Rezept stammt wieder aus der Zeitschrift „slowly veggie“, wobei ich die Suppe dieses Mal etwas verändert zubereitet habe. Aber die Grundidee ist geblieben!

Die Suppe ist also dementsprechend auch vegetarisch, was bei Suppe nicht selbstverständlich ist, da viele Suppen Hühnerfond als Basis nutzen, den ich auch sehr gerne und oft verwende. Bei der Pastinaken-Suppe kommt Gemüsefond zum Einsatz und das passt auch ganz wunderbar. Wer die Brösel nicht in Butter, sondern in Margarine oder Öl braten möchte, bereitet die Suppe dann sogar vegan zu.\footnote{Maltes Kitchen: \url{https://www.malteskitchen.de/pastinaken-mandel-suppe-mit-pumpernickelbroesel/}}

%%%%%%%%%%%%%%%%%%%%%%%%%%%%%%%%%%%%%%%%%%%%%%%%%%%%%%%%%%%%%%%%%%%%%%%%%%%%%%%%%%%%%
\begin{recipe}[rec:pastinaken-mandel-suppe]{Pastinaken-Mandel-Suppe}{4 Personen}{1\fr12 h}
    \freeform Diese cremige Pastinaken-Mandel-Suppe wäre ganz sicher eine schöne Vorspeise in einem weihnachtlichem Menü. Sie überzeugt nicht nur geschmacklich auf ganzer Linie, sondern wird auch optisch sehr schön mit Pumpernickel-Brösel und Mandelblättchen angerichtet. Perfekt auch für kalte Herbsttage!

    \ing[450]{g}{Pastinaken}
    \ing[2]{}{Zwiebeln, klein}
    \ing[2-3]{EL}{Olivenöl}
    \ing[150]{ml}{Weißwein}
    Zwiebel und Pastinaken schälen und würfeln.

    Einen großen Topf auf mittlerer Hitze aufsetzen und die Zwiebelwürfel sowie die Pastinakenwürfel im Olivenöl einige Minuten anschwitzen. Den Weißwein angießen und komplett reduzieren lassen, bis die Pastinaken- und Zwiebelwürfel anfangen, am Topfboden anzusetzen.

    \ing[250]{ml}{Sahne oder Sojasahne}
    \ing[600]{ml}{Gemüsefond}
    \ing[2]{EL}{Mandelmus}
    Die Sahne und den Fond angießen, das Mandelmus unterrühren und die Suppe zugedeckt \SIrange{15}{20}{\minute} köcheln lassen, bis die Pastinaken weich sind.

    \ing[2]{EL}{Meerrettich, gerieben aus dem Glas}
    \ing[2]{EL}{Apfelessig}
    \ing[]{}{Pfeffer \& Salz}
    Die Suppe vom Herd nehmen und mit Meerrettich, Essig, weißem Pfeffer sowie Salz abschmecken und in der Küchenmaschine sehr fein pürieren.

    \ing[1]{handvoll}{Mandelblättchen, geröstet}
    Die Suppe zusammen mit den Pumpernickel-Brösel (Rezept~\ref{rec:pumpernickel-brösel}) und Mandelblättchen servieren.
    \freeform\hrulefill
\end{recipe}

%%%%%%%%%%%%%%%%%%%%%%%%%%%%%%%%%%%%%%%%%%%%%%%%%%%%%%%%%%%%%%%%%%%%%%%%%%%%%%%%%%%%%
\begin{recipe}[rec:pumpernickel-brösel]{Pumpernickel-Brösel}{}{}
    \ing[1]{Scheibe}{Pumpernickel}
    \ing[30]{g}{Butter}
    \ing[1-2]{EL}{Petersilie, fein gehackt}
    \ing[]{}{Salz}
    Die Scheibe Pumpernickel fein zerbröseln, die Butter in einer kleinen Pfanne leicht bräunen und die Pumpernickel darin \SIrange{4}{5}{\minute} anbraten. Vom Herd nehmen, die gehackte Petersilie untermischen und die Brösel mit etwas Salz würzen.
    \freeform\hrulefill
\end{recipe}
